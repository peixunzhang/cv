\documentclass[10pt,a4paper,sans]{moderncv}
\moderncvstyle{custom}
\moderncvcolor{blue}

\usepackage[utf8]{inputenc}
\usepackage[scale=0.9]{geometry}
\usepackage{graphicx}
\graphicspath{{graphics/}}

\name{Maxim}{Schuwalow}
\address{Trusetaler Straße 92}{12687 Berlin}{Germany}
\email{maxim.schuwalow@gmail.com}
\social[linkedin]{mschuwalow}
\social[github]{mschuwalow}
\phone[mobile]{+49~(1520)~3154702}
\photo[80pt][0.4pt]{picture}

\begin{document}
\makecvtitle

\section{Education}
\cventry{October 2015 -- present}
        {B.Sc Cognitive Science}
        {Universität Osnabrück}
        {Osnabrück}
        {}
        {Expected Final Grade: \emph{1.2}}
\cventry{August 2006 -- July 2015}
        {Abitur}
        {Johannes-Brahms-Gymnasium}
        {Pinneberg}
        {}
        {Final Grade: \emph{1.3}}

\section{Experience}
\cventry{October 2019 -- present}
        {Software Engineer}
        {Zalando SE}
        {Berlin}
        {}
        {Implementation of the pipeline responsible for onboarding partner articles into the Zalando catalog.
         \newline Scala/Haskell microservices communicating using Zalando's Nakadi eventbus.}

\cventry{February 2017 -- October 2019}
        {Data Engineer}
        {Agile IM}
        {Osnabrück}
        {}
        {Streaming analytics / machine learning pipelines using Apache Kafka and Flink.
         \newline Hybrid Scala / Python machine learning applications using Py4J and Flink.
         \newline Microservices in Python and Scala, running on k8s.
         \newline Time series modeling and forecasting (Random Forests, LSTMs, MLPs).}

\section{Other Projects}
\cventry{December 2019}
        {Functional Scala 2019}
        {}
        {London}
        {}
        {Talk about automatically deriving intersection types in Scala using macros.
         \newline Slides can be found at \httplink[github.com/mschuwalow/functional-scala-2019]{https://github.com/mschuwalow/functional-scala-2019/blob/master/2019-functional-scala.pdf}.}

\cventry{May 2017}
        {WeAreDevelopers 2017}
        {}
        {Vienna}
        {}
        {Talk about using machine learning techniques to automatically classify GitHub repositories.}

\cventry{November 2016 - March 2017}
        {\httplink[InformatiCup2017]{https://github.com/InformatiCup/InformatiCup2017}}
        {German Association for Computer Science}
        {}
        {}
        {Classification of GitHub repositories using Deep Learning.\newline 4th place in Germany-wide competition.
         \newline Code can be found at \httplink[github.com/toludwig/golden-lemurs] {github.com/toludwig/golden-lemurs}.}

\section{Programming Languages}
\cvitem {Scala} {
  \begin{itemize}
    \item Good knowledge of the functional ecosystem.
    \item Contributions to:
    \begin{itemize}
        \item ZIO and ecosystem
        \item cats ecosystem
        \vspace{-0.3cm}
    \end{itemize}
  \end{itemize}
}

\cvitem {Haskell} {
  \begin{itemize}
      \item Intermediate knowledge of the language and common libraries.
      \vspace{-0.3cm}
  \end{itemize}
}

\cvitem {Python} {
  \begin{itemize}
    \item Knowledge of most common machine learning frameworks.
    \item Some experience writing Cython C-extensions.
    \vspace{-0.3cm}
  \end{itemize}
}

\section{Languages}
\cvitem{English}{fluent -- Business Level.}
\cvitem{German}{native.}

%=========================Signature=============================
\emptysection{}\closesection
\vfill
\vspace{0.5cm}
\begin{minipage}{0.6\textwidth}
\begin{center}\vspace{-0.1cm}\hspace{-1.3cm}
\phantom{\includegraphics[width=0.4\textwidth]{signature}}\\
\end{center}
\end{minipage}
\begin{minipage}{0.4\textwidth}
\begin{center}\vspace{-0.1cm}\hspace{-1.3cm}
\includegraphics[width=0.8\textwidth]{signature}\\
\end{center}
\vspace{-0.5cm}
\hspace{1.1cm} Maxim~Schuwalow, \today
\vspace{0.5cm}
\end{minipage}

\end{document}
